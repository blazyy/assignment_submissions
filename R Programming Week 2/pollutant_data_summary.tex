% Options for packages loaded elsewhere
\PassOptionsToPackage{unicode}{hyperref}
\PassOptionsToPackage{hyphens}{url}
%
\documentclass[
]{article}
\usepackage{lmodern}
\usepackage{amssymb,amsmath}
\usepackage{ifxetex,ifluatex}
\ifnum 0\ifxetex 1\fi\ifluatex 1\fi=0 % if pdftex
  \usepackage[T1]{fontenc}
  \usepackage[utf8]{inputenc}
  \usepackage{textcomp} % provide euro and other symbols
\else % if luatex or xetex
  \usepackage{unicode-math}
  \defaultfontfeatures{Scale=MatchLowercase}
  \defaultfontfeatures[\rmfamily]{Ligatures=TeX,Scale=1}
\fi
% Use upquote if available, for straight quotes in verbatim environments
\IfFileExists{upquote.sty}{\usepackage{upquote}}{}
\IfFileExists{microtype.sty}{% use microtype if available
  \usepackage[]{microtype}
  \UseMicrotypeSet[protrusion]{basicmath} % disable protrusion for tt fonts
}{}
\makeatletter
\@ifundefined{KOMAClassName}{% if non-KOMA class
  \IfFileExists{parskip.sty}{%
    \usepackage{parskip}
  }{% else
    \setlength{\parindent}{0pt}
    \setlength{\parskip}{6pt plus 2pt minus 1pt}}
}{% if KOMA class
  \KOMAoptions{parskip=half}}
\makeatother
\usepackage{xcolor}
\IfFileExists{xurl.sty}{\usepackage{xurl}}{} % add URL line breaks if available
\IfFileExists{bookmark.sty}{\usepackage{bookmark}}{\usepackage{hyperref}}
\hypersetup{
  pdftitle={Pollutant Data Summary},
  pdfauthor={Faaez Razeen},
  hidelinks,
  pdfcreator={LaTeX via pandoc}}
\urlstyle{same} % disable monospaced font for URLs
\usepackage[margin=1in]{geometry}
\usepackage{color}
\usepackage{fancyvrb}
\newcommand{\VerbBar}{|}
\newcommand{\VERB}{\Verb[commandchars=\\\{\}]}
\DefineVerbatimEnvironment{Highlighting}{Verbatim}{commandchars=\\\{\}}
% Add ',fontsize=\small' for more characters per line
\usepackage{framed}
\definecolor{shadecolor}{RGB}{248,248,248}
\newenvironment{Shaded}{\begin{snugshade}}{\end{snugshade}}
\newcommand{\AlertTok}[1]{\textcolor[rgb]{0.94,0.16,0.16}{#1}}
\newcommand{\AnnotationTok}[1]{\textcolor[rgb]{0.56,0.35,0.01}{\textbf{\textit{#1}}}}
\newcommand{\AttributeTok}[1]{\textcolor[rgb]{0.77,0.63,0.00}{#1}}
\newcommand{\BaseNTok}[1]{\textcolor[rgb]{0.00,0.00,0.81}{#1}}
\newcommand{\BuiltInTok}[1]{#1}
\newcommand{\CharTok}[1]{\textcolor[rgb]{0.31,0.60,0.02}{#1}}
\newcommand{\CommentTok}[1]{\textcolor[rgb]{0.56,0.35,0.01}{\textit{#1}}}
\newcommand{\CommentVarTok}[1]{\textcolor[rgb]{0.56,0.35,0.01}{\textbf{\textit{#1}}}}
\newcommand{\ConstantTok}[1]{\textcolor[rgb]{0.00,0.00,0.00}{#1}}
\newcommand{\ControlFlowTok}[1]{\textcolor[rgb]{0.13,0.29,0.53}{\textbf{#1}}}
\newcommand{\DataTypeTok}[1]{\textcolor[rgb]{0.13,0.29,0.53}{#1}}
\newcommand{\DecValTok}[1]{\textcolor[rgb]{0.00,0.00,0.81}{#1}}
\newcommand{\DocumentationTok}[1]{\textcolor[rgb]{0.56,0.35,0.01}{\textbf{\textit{#1}}}}
\newcommand{\ErrorTok}[1]{\textcolor[rgb]{0.64,0.00,0.00}{\textbf{#1}}}
\newcommand{\ExtensionTok}[1]{#1}
\newcommand{\FloatTok}[1]{\textcolor[rgb]{0.00,0.00,0.81}{#1}}
\newcommand{\FunctionTok}[1]{\textcolor[rgb]{0.00,0.00,0.00}{#1}}
\newcommand{\ImportTok}[1]{#1}
\newcommand{\InformationTok}[1]{\textcolor[rgb]{0.56,0.35,0.01}{\textbf{\textit{#1}}}}
\newcommand{\KeywordTok}[1]{\textcolor[rgb]{0.13,0.29,0.53}{\textbf{#1}}}
\newcommand{\NormalTok}[1]{#1}
\newcommand{\OperatorTok}[1]{\textcolor[rgb]{0.81,0.36,0.00}{\textbf{#1}}}
\newcommand{\OtherTok}[1]{\textcolor[rgb]{0.56,0.35,0.01}{#1}}
\newcommand{\PreprocessorTok}[1]{\textcolor[rgb]{0.56,0.35,0.01}{\textit{#1}}}
\newcommand{\RegionMarkerTok}[1]{#1}
\newcommand{\SpecialCharTok}[1]{\textcolor[rgb]{0.00,0.00,0.00}{#1}}
\newcommand{\SpecialStringTok}[1]{\textcolor[rgb]{0.31,0.60,0.02}{#1}}
\newcommand{\StringTok}[1]{\textcolor[rgb]{0.31,0.60,0.02}{#1}}
\newcommand{\VariableTok}[1]{\textcolor[rgb]{0.00,0.00,0.00}{#1}}
\newcommand{\VerbatimStringTok}[1]{\textcolor[rgb]{0.31,0.60,0.02}{#1}}
\newcommand{\WarningTok}[1]{\textcolor[rgb]{0.56,0.35,0.01}{\textbf{\textit{#1}}}}
\usepackage{graphicx,grffile}
\makeatletter
\def\maxwidth{\ifdim\Gin@nat@width>\linewidth\linewidth\else\Gin@nat@width\fi}
\def\maxheight{\ifdim\Gin@nat@height>\textheight\textheight\else\Gin@nat@height\fi}
\makeatother
% Scale images if necessary, so that they will not overflow the page
% margins by default, and it is still possible to overwrite the defaults
% using explicit options in \includegraphics[width, height, ...]{}
\setkeys{Gin}{width=\maxwidth,height=\maxheight,keepaspectratio}
% Set default figure placement to htbp
\makeatletter
\def\fps@figure{htbp}
\makeatother
\setlength{\emergencystretch}{3em} % prevent overfull lines
\providecommand{\tightlist}{%
  \setlength{\itemsep}{0pt}\setlength{\parskip}{0pt}}
\setcounter{secnumdepth}{-\maxdimen} % remove section numbering

\title{Pollutant Data Summary}
\author{Faaez Razeen}
\date{2/26/2020}

\begin{document}
\maketitle

\hypertarget{pollutant-data}{%
\subsection{Pollutant Data}\label{pollutant-data}}

The zip file contains 332 comma-separated-value (CSV) files containing
pollution monitoring data for fine particulate matter (PM) air pollution
at 332 locations in the United States. Each file contains data from a
single monitor and the ID number for each monitor is contained in the
file name. For example, data for monitor 200 is contained in the file
``200.csv''. Each file contains three variables:

\begin{itemize}
\tightlist
\item
  Date: the date of the observation in YYYY-MM-DD format
  (year-month-day)
\item
  sulfate: the level of sulfate PM in the air on that date (measured in
  micrograms per cubic meter)
\item
  nitrate: the level of nitrate PM in the air on that date (measured in
  micrograms per cubic meter)
\end{itemize}

\begin{Shaded}
\begin{Highlighting}[]
\NormalTok{files <-}\StringTok{ }\KeywordTok{list.files}\NormalTok{(}\StringTok{'specdata'}\NormalTok{, }\DataTypeTok{full.names =} \OtherTok{TRUE}\NormalTok{)}
\NormalTok{df <-}\StringTok{ }\KeywordTok{data.frame}\NormalTok{()}
\ControlFlowTok{for}\NormalTok{(file }\ControlFlowTok{in}\NormalTok{ files)\{df <-}\StringTok{ }\KeywordTok{rbind}\NormalTok{(df, }\KeywordTok{read.csv}\NormalTok{(file))\}}
\KeywordTok{summary}\NormalTok{(df)}
\end{Highlighting}
\end{Shaded}

\begin{verbatim}
##          Date           sulfate          nitrate             ID       
##  2004-01-01:   250   Min.   : 0.0     Min.   : 0.0     Min.   :  1.0  
##  2004-01-02:   250   1st Qu.: 1.3     1st Qu.: 0.4     1st Qu.: 79.0  
##  2004-01-03:   250   Median : 2.4     Median : 0.8     Median :168.0  
##  2004-01-04:   250   Mean   : 3.2     Mean   : 1.7     Mean   :164.5  
##  2004-01-05:   250   3rd Qu.: 4.0     3rd Qu.: 2.0     3rd Qu.:247.0  
##  2004-01-06:   250   Max.   :35.9     Max.   :53.9     Max.   :332.0  
##  (Other)   :770587   NA's   :653304   NA's   :657738
\end{verbatim}

\hypertarget{date-range}{%
\subsubsection{Date Range}\label{date-range}}

\begin{Shaded}
\begin{Highlighting}[]
\NormalTok{start_date <-}\StringTok{ }\KeywordTok{as.Date}\NormalTok{(df[}\DecValTok{1}\NormalTok{, ][[}\StringTok{'Date'}\NormalTok{]])}
\NormalTok{end_date <-}\StringTok{ }\KeywordTok{as.Date}\NormalTok{(df[}\KeywordTok{dim}\NormalTok{(df)[}\DecValTok{1}\NormalTok{], ][[}\StringTok{'Date'}\NormalTok{]])}
\KeywordTok{paste}\NormalTok{(}\StringTok{"Values ranging from"}\NormalTok{, start_date, }\StringTok{"to"}\NormalTok{, end_date, }\StringTok{"with a time difference of "}\NormalTok{, end_date }\OperatorTok{-}\StringTok{ }\NormalTok{start_date,}\StringTok{"days"}\NormalTok{)}
\end{Highlighting}
\end{Shaded}

\begin{verbatim}
## [1] "Values ranging from 2003-01-01 to 2004-12-31 with a time difference of  730 days"
\end{verbatim}

\hypertarget{missing-values}{%
\subsubsection{Missing Values}\label{missing-values}}

Has a lot of missing values.

\begin{Shaded}
\begin{Highlighting}[]
\NormalTok{missing_percent <-}\StringTok{ }\NormalTok{(}\DecValTok{1} \OperatorTok{-}\StringTok{ }\NormalTok{((}\KeywordTok{dim}\NormalTok{(df[}\KeywordTok{complete.cases}\NormalTok{(df), ])[}\DecValTok{1}\NormalTok{])}\OperatorTok{/}\NormalTok{(}\KeywordTok{dim}\NormalTok{(df)[}\DecValTok{1}\NormalTok{]))) }\OperatorTok{*}\StringTok{ }\DecValTok{100}
\KeywordTok{paste}\NormalTok{(}\StringTok{"Percentage of missing data:"}\NormalTok{, }\KeywordTok{round}\NormalTok{(missing_percent, }\DataTypeTok{digits =} \DecValTok{2}\NormalTok{))}
\end{Highlighting}
\end{Shaded}

\begin{verbatim}
## [1] "Percentage of missing data: 85.52"
\end{verbatim}

\end{document}
